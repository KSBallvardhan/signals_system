%iffalse
\let\negmedspace\undefined
\let\negthickspace\undefined
\documentclass[journal,12pt,twocolumn]{IEEEtran}
\usepackage{cite}
\usepackage{amsmath,amssymb,amsfonts}
\usepackage{graphicx}
\usepackage{textcomp}
\usepackage{xcolor}
\usepackage{txfonts}
\usepackage{listings}
\usepackage{enumitem}
\usepackage{mathtools}
\usepackage{gensymb}
\usepackage{comment}
\usepackage[breaklinks=true]{hyperref}
\usepackage{tkz-euclide} 
\usepackage{listings}
\usepackage{gvv}                                        
\def\inputGnumericTable{}                                 
\usepackage[latin1]{inputenc}                                
\usepackage{color}                                            
\usepackage{array}                                            
\usepackage{longtable}                                       
\usepackage{calc}                                             
\usepackage{multirow}                                         
\usepackage{hhline}                                           
\usepackage{ifthen}                                           
\usepackage{lscape}
\usepackage[export]{adjustbox}

\newtheorem{theorem}{Theorem}[section]
\newtheorem{problem}{Problem}
\newtheorem{proposition}{Proposition}[section]
\newtheorem{lemma}{Lemma}[section]
\newtheorem{corollary}[theorem]{Corollary}
\newtheorem{example}{Example}[section]
\newtheorem{definition}[problem]{Definition}
\newcommand{\BEQA}{\begin{eqnarray}}
	\newcommand{\EEQA}{\end{eqnarray}}
\newcommand{\define}{\stackrel{\triangle}{=}}
\newtheorem{rem}{Remark}

\begin{document}
	\parindent 0px
	\bibliographystyle{IEEEtran}
	
	\vspace{3cm}
	
	\title{}
	\author{EE23BTECH11209 - K S Ballvardhan$^{*}$
	}
	\maketitle
	\newpage
	\bigskip
	
	% \renewcommand{\thefigure}{\theenumi}
	% \renewcommand{\thetable}{\theenumi}
	
	
	\section*{Chapter 15}
	
	\noindent \textbf{17.} A pipe 20 cm long is closed at one end. Which harmonic mode of the pipe is resonantly excited by a 430 Hz source ? Will the same source be in resonance with
	the pipe if both ends are open? (speed of sound in air is 340 m $s^{-1}$).\\
	
	\textbf {Solution: }
	
	\begin{table}[ht] 
		\centering
		\input{tables/table1}
		\caption{input values}
		\label{tab: Table analog1}
	\end{table}
	Fundamental frequency of a pipe with one closed end is given by,
	\begin{align}
		f_c\brak{o} &= \frac{v}{4l}
	\end{align}
	where, v is speed of sound and l is length of the tube.
	$ n^{th}$ harmonic frequency of closed-end pipe is given by,
	\begin{align}
		f_c\brak{n} &= \frac{v\brak{2n+1}}{4l}\quad n= 1,2,\dots
	\end{align}
	From \tabref{tab: Table analog1}:
	\begin{align}
		\frac{v\brak{2n-1}}{4l} &= F\\
		2n+1 &= \frac{F\brak{4l}}{v}\\
		\implies 2n-1 &\approx 1.01\\
		\therefore n&\approx 1.005
	\end{align}
	For the Fundamental frequency, the open pipe and source are in resonance.\\\\
	If the pipe is open at both ends, its fundamental frequency is given by,
	\begin{align}
		f_c\brak{o} &= \frac{v}{2l}
	\end{align}
	$ n^{th}$ harmonic frequency of open-end pipe is given by,
	\begin{align}
		f_o\brak{n} &= \frac{v\brak{n}}{2l}\quad n= 1,2,\dots
	\end{align}
	From \tabref{tab: Table analog1}:
	\begin{align}
		\frac{v\brak{n}}{2l} &= F\\
		n &= \frac{F\brak{2l}}{v}\\
		\implies n &\approx 0.5
	\end{align}
	So, source and the same pipe with two open ends can never be in resonance.
\end{document}
